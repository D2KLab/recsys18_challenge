\section{Conclusion and Future Work}
\label{sec:conclusion}
Completing automatically playlists with tracks contained in the MPD dataset is a particularly difficult task due to the dataset dimension and the variety of playlists generated by numerous users having different likes and behaviors bringing great diversity. In this paper, we present the D2KLab recommender system that implements an ensemble approach of multiple learning models differently optimized combined with a Borda count strategy. Each model runs an RNN that exploits a wide range of playlist features such as artist, album, track, lyrics (used for the creative track), title and a so-called Title2Rec that takes as input the title and that is used, as fall-back strategy, when playlists do not contain any track. The approach showed to be robust in such a complex setting demonstrating the effectiveness of learning models for automatic playlist completion.

The experimental analysis brought to further attention three points, namely the generation strategy, complementarity of the learning models, and computing time. The generation strategy has a great impact on the results and it pointed out that a recurrent decoding stage is less performing than using a ranking strategy that weighs the output of each RNN of the encoding stage. The ensemble strategy aggregates different outputs of the learning model runs by pivoting the generated ranking. This has granted a sensible increment in performance, so we plan to study further the complementarity of the runs and to build a learning model to automatically select the best candidates. Finally, the computing time has been a crucial experimental setup element due to the generation of the RNN learning model; we addressed it by creating different sizes of the MPD dataset randomly selected and by optimizing the learning models on the hardware a disposal, becoming another factor of differentiation for shaping a performing submission.


%In this paper we presented a novel approach for music recommendation built on top of the Million Playlist Dataset. The strategy involves 3 types of vectors -- Sequential embeddings, Titles embeddings and Lyrics embeddings -- that are used for train a RNN based on LSTM. The final recommendation comes from an ensemble architecture the combine the results of an RNN and the ones of Title2Rec, a playlist generation algorithm that relies on the sole title of the playlist.