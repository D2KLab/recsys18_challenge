\section{Introduction}
\label{sec:introduction}
In recent years, music streaming services strongly modified the way in which people access to music content. In particular, the music experience does not foresee anymore to follow pre-defined collections of tracks (albums) edited by music stakeholders (artists and labels): the end-user can now produce her/his own playlist with potentially unlimited freedom. As a consequence, the automatic playlist generation and continuation are now crucial tasks in the recommender system field.

This paper describes our results for the task of playlist completion obtained in the context of the RecSys Challenge 2018. This work relies on an ensemble strategy which involves different types of features, including sequential embeddings, title embeddings and lyrics features.\footnote{The lyrics have been used only for the Creative Track.} Following the challenge rules,\footnote{\url{https://recsys-challenge.spotify.com/rules}} the target dataset is the Million Playlist Dataset (MPD), which contains metadata for 1 million playlists gathering more than 2.2 million distinct tracks. The implementation of our approach is publicly available at \url{https://github.com/D2KLab/recsys18_challenge}.

The remainder of the paper is structured as follows: Section~\ref{sec:ensemble} presents our ensemble approach, while Section~\ref{sec:rnn} details the design of the Recurrent Neural Networks. In Section~\ref{sec:t2r} we discuss the intuition behind the implementation of Title2Rec. Section~\ref{sec:optimization} explains the optimization conducted on the ensemble, the RNN, and Title2Rec. We describe the experimental results in Section~\ref{sec:results}. Finally, in Section~\ref{sec:conclusion} we provide the conclusions.